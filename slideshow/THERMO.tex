% Options for packages loaded elsewhere
\PassOptionsToPackage{unicode}{hyperref}
\PassOptionsToPackage{hyphens}{url}
%
\documentclass[
  ignorenonframetext,
]{beamer}
\usepackage{pgfpages}
\setbeamertemplate{caption}[numbered]
\setbeamertemplate{caption label separator}{: }
\setbeamercolor{caption name}{fg=normal text.fg}
\beamertemplatenavigationsymbolsempty
% Prevent slide breaks in the middle of a paragraph
\widowpenalties 1 10000
\raggedbottom
\setbeamertemplate{part page}{
  \centering
  \begin{beamercolorbox}[sep=16pt,center]{part title}
    \usebeamerfont{part title}\insertpart\par
  \end{beamercolorbox}
}
\setbeamertemplate{section page}{
  \centering
  \begin{beamercolorbox}[sep=12pt,center]{part title}
    \usebeamerfont{section title}\insertsection\par
  \end{beamercolorbox}
}
\setbeamertemplate{subsection page}{
  \centering
  \begin{beamercolorbox}[sep=8pt,center]{part title}
    \usebeamerfont{subsection title}\insertsubsection\par
  \end{beamercolorbox}
}
\AtBeginPart{
  \frame{\partpage}
}
\AtBeginSection{
  \ifbibliography
  \else
    \frame{\sectionpage}
  \fi
}
\AtBeginSubsection{
  \frame{\subsectionpage}
}
\usepackage{lmodern}
\usepackage{amssymb,amsmath}
\usepackage{ifxetex,ifluatex}
\ifnum 0\ifxetex 1\fi\ifluatex 1\fi=0 % if pdftex
  \usepackage[T1]{fontenc}
  \usepackage[utf8]{inputenc}
  \usepackage{textcomp} % provide euro and other symbols
\else % if luatex or xetex
  \usepackage{unicode-math}
  \defaultfontfeatures{Scale=MatchLowercase}
  \defaultfontfeatures[\rmfamily]{Ligatures=TeX,Scale=1}
\fi
% Use upquote if available, for straight quotes in verbatim environments
\IfFileExists{upquote.sty}{\usepackage{upquote}}{}
\IfFileExists{microtype.sty}{% use microtype if available
  \usepackage[]{microtype}
  \UseMicrotypeSet[protrusion]{basicmath} % disable protrusion for tt fonts
}{}
\makeatletter
\@ifundefined{KOMAClassName}{% if non-KOMA class
  \IfFileExists{parskip.sty}{%
    \usepackage{parskip}
  }{% else
    \setlength{\parindent}{0pt}
    \setlength{\parskip}{6pt plus 2pt minus 1pt}}
}{% if KOMA class
  \KOMAoptions{parskip=half}}
\makeatother
\usepackage{xcolor}
\IfFileExists{xurl.sty}{\usepackage{xurl}}{} % add URL line breaks if available
\IfFileExists{bookmark.sty}{\usepackage{bookmark}}{\usepackage{hyperref}}
\hypersetup{
  pdftitle={Rappels de thermodynamique},
  pdfauthor={Vincent Le Chenadec},
  hidelinks,
  pdfcreator={LaTeX via pandoc}}
\urlstyle{same} % disable monospaced font for URLs
\newif\ifbibliography
\usepackage{longtable,booktabs}
\usepackage{caption}
% Make caption package work with longtable
\makeatletter
\def\fnum@table{\tablename~\thetable}
\makeatother
\setlength{\emergencystretch}{3em} % prevent overfull lines
\providecommand{\tightlist}{%
  \setlength{\itemsep}{0pt}\setlength{\parskip}{0pt}}
\setcounter{secnumdepth}{5}
\usepackage{siunitx}
\usepackage{amsmath}
\usepackage{stmaryrd}

\title{Rappels de thermodynamique}
\author{Vincent Le Chenadec}
\date{MFT-3-1-2 2021/2022}

\begin{document}
\frame{\titlepage}

\begin{frame}{Fonctions homogènes}
\protect\hypertarget{fonctions-homoguxe8nes}{}
\begin{block}{Définition}
\protect\hypertarget{duxe9finition}{}
Une fonction \(f \colon \mathbb{R} ^ n \to \mathbb{R} ^ m\) est dite
positivement homogène de degré \(k\) si \[
\forall \lambda > 0, \quad f \left ( \lambda x _ 1, \ldots, \lambda x _ n \right ) = \lambda ^ k f \left ( x _ 1, \ldots, x _ n \right ).
\]
\end{block}

\begin{block}{Propriété}
\protect\hypertarget{propriuxe9tuxe9}{}
Si \(f\) est différentiable, alors ces dérivées partielles premières
sont homogènes de degré \(k - 1\).
\end{block}
\end{frame}

\begin{frame}
\begin{block}{Identité d'Euler}
\protect\hypertarget{identituxe9-deuler}{}
Une fonction de plusieurs variables
\(f \colon \mathbb{R} ^ n \to \mathbb{R} ^ m\) différentiable en tout
point est positivement homogène de degré \(k\) si et seulement si la
relation suivante, appelée identité d'Euler, est vérifiée \[
\forall x = \left ( x_1, \ldots, x _ n \right ) \in \mathbb{R} ^ n, \quad \sum _ {i = 1} ^ n x _ i \left ( \frac{\partial f}{\partial x _ i} \right ) _ {x _ {j \ne i}} = k f \left ( x \right ).
\]

\begin{itemize}
\tightlist
\item
  Une grandeur intensive correspond à \(k = 0\) ;
\item
  Une grandeur extensive à \(k = 1\).
\end{itemize}
\end{block}
\end{frame}

\begin{frame}{Grandeurs d'état}
\protect\hypertarget{grandeurs-duxe9tat}{}
\begin{itemize}
\item
  La pression : \(p\) (exprimée en \(\si{\pascal}\))
\item
  La température : \(T\) (exprimée en \(\si{\kelvin}\))
\item
  Les quantités de matière (exprimées en \(\si{\mole}\))

  \begin{itemize}
  \tightlist
  \item
    De l'espèce \(i \in \left \llbracket 1, N \right \rrbracket\) :
    \(n _ i\)
  \item
    Totale : \(n = \sum _ {i = 1} ^ N n _ i\)
  \end{itemize}
\item
  Le volume molaire \(V\) (exprimé en \(\si{\meter\cubed\per\mole}\))
\item
  L'énergie interne molaire \(U\) (exprimée en \(\si{\joule\per\mole}\))
\item
  L'entropie molaire \(S\) (exprimée en
  \(\si{\joule\per\kelvin\per\mole}\))
\item
  \(H\), \(F\)\ldots{}
\end{itemize}
\end{frame}

\begin{frame}{Fonctions d'état usuelles}
\protect\hypertarget{fonctions-duxe9tat-usuelles}{}
\begin{itemize}
\tightlist
\item
  Énergie interne molaire \[
  \left ( n U \right ) = T \left ( n S \right ) - p \left ( n V \right ) + \sum _ {i = 1} ^ N \left ( n _ i \mu _ i \right )
  \]
\item
  Énergie libre molaire (ou énergie de Helmholtz) \[
  \left ( n F \right ) = \left ( n U \right ) - T \left ( n S \right ) = - p \left ( n V \right ) + \sum _ {i = 1} ^ N \left ( n _ i \mu _ i \right )
  \]
\item
  Enthalpie molaire \[
  \left ( n H \right ) = \left ( n U \right ) + p \left ( n V \right ) = T \left ( n S \right ) + \sum _ {i = 1} ^ N \left ( n _ i \mu _ i \right )
  \]
\item
  Enthalpie libre molaire (ou énergie de Gibbs) \[
  \left ( n G \right ) = \left ( n H \right ) - T \left ( n S \right ) = \sum _ {i = 1} ^ N \left ( n _ i \mu _ i \right )
  \]
\end{itemize}
\end{frame}

\begin{frame}{Convexité et transformation de Legendre}
\protect\hypertarget{convexituxe9-et-transformation-de-legendre}{}
\begin{itemize}
\tightlist
\item
  Variables conjuguées
\end{itemize}

\begin{longtable}[]{@{}cc@{}}
\toprule
Grandeur extensive & Grandeur intensive\tabularnewline
\midrule
\endhead
\(\left ( n V \right )\) & \(-p\)\tabularnewline
\(\left ( n S \right )\) & \(T\)\tabularnewline
\(n _ i\) & \(\mu _ i\)\tabularnewline
\bottomrule
\end{longtable}

\begin{itemize}
\tightlist
\item
  Variables naturelles
\end{itemize}

\begin{longtable}[]{@{}cc@{}}
\toprule
Grandeur d'état & Variables naturelles\tabularnewline
\midrule
\endhead
\(\left ( n U \right )\) &
\(\left ( V, n S, n _ 1, \ldots, n _ N \right )\)\tabularnewline
\(\left ( n H \right )\) &
\(\left ( p, n S, n _ 1, \ldots, n _ N \right )\)\tabularnewline
\(\left ( n F \right )\) &
\(\left ( V, T, n _ 1, \ldots, n _ N \right )\)\tabularnewline
\(\left ( n G \right )\) &
\(\left ( p, T, n _ 1, \ldots, n _ N \right )\)\tabularnewline
\bottomrule
\end{longtable}
\end{frame}

\begin{frame}{Identités fondamentales}
\protect\hypertarget{identituxe9s-fondamentales}{}
\begin{itemize}
\tightlist
\item
  Énergie interne \[
  \mathrm{d} \left ( n U \right ) = T \mathrm{d} \left ( n S \right ) - p \mathrm{d} \left ( n V \right ) + \sum _ {i = 1} ^ N \mu _ i \mathrm{d} n _ i
  \]
\item
  Énergie libre \[
  \mathrm{d} \left ( n F \right ) = -\left ( n S \right ) \mathrm{d} T - p \mathrm{d} \left ( n V \right ) + \sum _ {i = 1} ^ N \mu _ i \mathrm{d} n _ i
  \]
\item
  Enthalpie \[
  \mathrm{d} \left ( n H \right ) = T \mathrm{d} \left ( n S \right ) + \left ( n V \right ) \mathrm{d} p + \sum _ {i = 1} ^ N \mu _ i \mathrm{d} n _ i
  \]
\item
  Enthalpie libre \[
  \mathrm{d} \left ( n G \right ) = -\left ( n S \right ) \mathrm{d} T + \left ( n V \right ) \mathrm{d} p + \sum _ {i = 1} ^ N \mu _ i \mathrm{d} n _ i
  \] \{\#eq:fondamentalgibbs\}
\end{itemize}
\end{frame}

\begin{frame}{Relation de Gibbs-Duhem}
\protect\hypertarget{relation-de-gibbs-duhem}{}
\[
\left ( n V \right ) \mathrm{d} p - \left ( n S \right ) \mathrm{d} T = \sum _ {i = 1} ^ N n _ i \mathrm{d} \mu _ i.
\] \{\#eq:gibbsduhem\}
\end{frame}

\begin{frame}{Grandeurs molaires pures}
\protect\hypertarget{grandeurs-molaires-pures}{}
\begin{itemize}
\tightlist
\item
  Lorsque un seul constituant (\(i\)) rentre dans la composition d'un
  milieu, on parlera de substance pure.
\item
  On notera \(U _ i\), \(G _ i\), \ldots, les grandeurs molaires
  associées à cette substance pure.
\item
  On aura notamment l'énergie interne molaire \[
  U _ i \colon \left \vert
  \begin{aligned}
  \mathbb{R} ^ 2 & \to \mathbb{R}, \\
  \left ( S, V \right ) & \mapsto U _ i \left ( S, V \right ).
  \end{aligned} \right .
  \]
\end{itemize}
\end{frame}

\begin{frame}{Grandeurs molaires partielles}
\protect\hypertarget{grandeurs-molaires-partielles}{}
\begin{itemize}
\tightlist
\item
  Pour toute grandeur molaire \(A\), on définit les grandeurs molaires
  partielles \[
  \overline{A} _ i \triangleq \left ( \frac{\partial \left ( n A \right )}{\partial n _ i} \right ) _ {p, T, n _ {j \ne i}}
  \]
\item
  Le volume molaire partiel est donc défini comme \[
  \overline{V} _ i \triangleq \left ( \frac{\partial \left ( n V \right )}{\partial n _ i} \right ) _ {p, T, n _ {j \ne i}}
  \]
\item
  Et l'entropie molaire partielle \[
  \overline{S} _ i \triangleq \left ( \frac{\partial \left ( n S \right )}{\partial n _ i} \right ) _ {p, T, n _ {j \ne i}}
  \]
\item
  On montre facilement que \[
  \overline{G} _ i \triangleq \left ( \frac{\partial \left ( n G \right )}{\partial n _ i} \right ) _ {p, T, n _ {j \ne i}} \triangleq \mu _ i
  \]
\item
  On définit enfin l'énergie interne partielle \[
  \overline{U} _ i \triangleq T \overline{S} _ i - p \overline{V} _ i + \mu _ i
  \] et l'enthalpie molaire partielle \[
  \overline{H} _ i = T \overline{S} _ i + \mu _ i
  \]
\end{itemize}
\end{frame}

\begin{frame}
\begin{itemize}
\tightlist
\item
  On montre (théorème d'Euler) \[
  \left ( n V \right ) = \sum _ {i = 1} ^ N n _ i \overline{V} _ i \quad \mathrm{et} \quad \left ( n S \right ) = \sum _ {i = 1} ^ N n _ i \overline{S} _ i
  \] \{\#eq:partialeuler\}
\item
  De telles relations existent pour toutes les grandeurs molaires
  partielles \[
  \left ( n A \right ) = \sum _ {i = 1} ^ N n _ i \overline{A} _ i \quad \mathrm{ou} \quad A = \sum _ {i = 1} ^ N x _ i \overline{A} _ i
  \]
\end{itemize}
\end{frame}

\begin{frame}
\begin{multline*}
\mathrm{d} \left ( n A \right ) = \left ( \frac{\partial \left ( n A \right )}{\partial p} \right ) _ {T, \mathbf{n}} \mathrm{d} p \\
\; + \left ( \frac{\partial \left ( n A \right )}{\partial T} \right ) _ {p, \mathbf{n}} \mathrm{d} T + \sum _ {i = 1} ^ N \left ( \frac{\partial \left ( n A \right )}{\partial n _ i} \right ) _ {p, T, n _ {j \ne i}} \mathrm{d} n _ i,
\end{multline*} donc en reconnaissant \(\overline{A} _ i\) dans le
dernier terme, et en notant que \[
\left ( \frac{\partial \left ( n A \right )}{\partial p} \right ) _ {T, \mathbf{n}} = n \left ( \frac{\partial \left ( A \right )}{\partial p} \right ) _ {T, \mathbf{x}}
\quad \mathrm{et} \quad
\left ( \frac{\partial \left ( n A \right )}{\partial T} \right ) _ {p, \mathbf{n}} = n \left ( \frac{\partial \left ( A \right )}{\partial T} \right ) _ {p, \mathbf{x}},
\] on trouve \begin{multline*}
n \left [ \mathrm{d} A - \left ( \frac{\partial A}{\partial p} \right ) _ {T, \mathbf{x}} \mathrm{d} p - \left ( \frac{\partial A}{\partial T} \right ) _ {p, \mathbf{x}} \mathrm{d} T - \sum _ {i = 1} ^ N \overline{A} _ i \mathrm{d} x _ i \right ] \\
\; + \left [ A - \sum _ {i = 1} ^ N x _ i \overline{A} _ i \right ] = 0.
\end{multline*}
\end{frame}

\begin{frame}
\begin{itemize}
\tightlist
\item
  Le dernier terme étant nul, on en déduit \[
  \mathrm{d} A = \left ( \frac{\partial A}{\partial p} \right ) _ {T, \mathbf{x}} \mathrm{d} p + \left ( \frac{\partial A}{\partial T} \right ) _ {p, \mathbf{x}} \mathrm{d} T + \sum _ {i = 1} ^ N \overline{A} _ i \mathrm{d} x _ i
  \]
\end{itemize}
\end{frame}

\begin{frame}
\begin{itemize}
\tightlist
\item
  En notant enfin que la différentielle de \[
  A = \sum _ {i = 1} ^ N x _ i \overline{A} _ i
  \] vaut \[
  \mathrm{d} A = \sum _ i x _ i \mathrm{d} \overline{A} _ i + \sum _ i \overline{A} _ i \mathrm{d} x_ i,
  \] on trouve en substituant dans la diapo précédente la relation de
  Gibbs-Duhem \[
   \sum _ i x _ i \mathrm{d} \overline{A} _ i = \left ( \frac{\partial A}{\partial p} \right ) _ {T, \mathbf{x}} \mathrm{d} p + \left ( \frac{\partial A}{\partial T} \right ) _ {p, \mathbf{x}} \mathrm{d} T
  \]
\end{itemize}
\end{frame}

\begin{frame}
\begin{itemize}
\tightlist
\item
  D'après la relation fondamentale @eq:fondamentalgibbs \[
  \overline{V} _ i = \left ( \frac{\partial \mu _ i}{\partial p} \right ) _ {T, n _ j} \quad \mathrm{et} \quad \overline{S} _ i = - \left ( \frac{\partial \mu _ i}{\partial T} \right ) _ {p, n _ j}
  \]
\item
  On en déduit \[
  \mathrm{d} \mu _ i = \overline{V} _ i \mathrm{d} p - \overline{S} _ i \mathrm{d} T + \sum _ {j = 1} ^ N \left ( \frac{\partial \mu _ i}{\partial x _ j} \right ) _ {p, T, n _ {k \ne j}} \mathrm{d} x _ i
  \]
\end{itemize}
\end{frame}

\begin{frame}{Densités}
\protect\hypertarget{densituxe9s}{}
\begin{itemize}
\tightlist
\item
  Concentration (ou densité molaire) de l'espèce \(i\) \[
  \forall i \in \left \llbracket 1, N \right \rrbracket, \quad c _ i = \frac{n _ i}{n V}
  \] et totale \[
  c = \sum _ {i = 1} ^ N c _ i
  \]
\item
  On déduit de @eq:partialeuler \[
  \sum _ {i = 1} ^ N c_ i \overline{V} _ i = \frac{n V}{n V} = 1 \quad \mathrm{et} \quad \sum _ {i = 1} ^ N c_i \overline{S} _ i = \frac{n S}{n V} = \frac{S}{V} = s
  \]
\end{itemize}
\end{frame}

\begin{frame}
\begin{itemize}
\tightlist
\item
  Masse volumique \[
  \forall i \in \left \llbracket 1, N \right \rrbracket, \quad \rho _ i = \frac{m _ i}{n V} = \frac{M _ i n _ i}{n V}
  \] où \(M _ i\) dénote la masse molaire
\item
  Volume spécifique \[
  v = \frac{1}{\rho}
  \]
\end{itemize}
\end{frame}

\begin{frame}
\begin{itemize}
\tightlist
\item
  La relation de Gibbs-Duhem (@eq:gibbsduhem) peut facilement se
  réécrire sous la forme \[
  \mathrm{d} p - s \mathrm{d} T = \sum _ {i = 1} ^ n c _ i \mathrm{d} \mu _ i.
  \]
\end{itemize}
\end{frame}

\begin{frame}{Fractions (ou titres)}
\protect\hypertarget{fractions-ou-titres}{}
\begin{itemize}
\tightlist
\item
  On définit les titres molaires comme suit \[
  \forall i \in \left \llbracket 1, N \right \rrbracket, \quad x _ i = \frac{c _ i}{c} = \frac{n _ i}{n}
  \]
\end{itemize}
\end{frame}

\begin{frame}{Équilibre}
\protect\hypertarget{uxe9quilibre}{}
\begin{itemize}
\item
  Deux systèmes \(A\) et \(B\) sont à l'équilibre

  \begin{itemize}
  \tightlist
  \item
    Mécanique lorsque \(p _ A = p _ B\) ;
  \item
    Thermique lorsque \(T _ A = T _ B\) ;
  \item
    Chimique lorsque \(\mu _ A = \mu _ B\).
  \end{itemize}
\item
  Enfin, on parlera également d'équilibre cinématique lorsque
  \(v _ A = v _ B\).
\end{itemize}
\end{frame}

\begin{frame}{Relation de Helmholtz}
\protect\hypertarget{relation-de-helmholtz}{}
\[
\left ( \frac{\partial U _ i}{\partial V} \right ) _ T = T ^ 2 \left [ \frac{\partial}{\partial T} \left ( \frac{p}{T} \right ) \right ] _ V
\]

\textbf{Démonstration}

\begin{enumerate}
\tightlist
\item
  Exprimer la différentielle de \(U _ i\) par rapport à \(V\) et \(T\).
\item
  Utiliser cette expression pour réécrire la relation fondamentale \[
  \mathrm{d} S _ i = \frac{1}{T} \mathrm{d} U _ i + \frac{p}{T} \mathrm{d} V.
  \] en fonction de \(\mathrm{d} V\) et \(\mathrm{d} T\).
\item
  Utiliser que la condition d'intégrabilité pour obtenir la relation de
  Helmholtz. \(S _ i \left ( U _ i, V \right )\) en fonction la
  différentielle de \(U _ i\) par rapport à \(V\) et \(T\).
\item
  Montrer que dans le modèle du gaz parfait, \(U _ i\) ne dépend pas du
  volume.
\item
  Qu'en est-il de l'équation d'état de van der Waals ?
\end{enumerate}
\end{frame}

\begin{frame}{Relation de Gibbs-Helmholtz}
\protect\hypertarget{relation-de-gibbs-helmholtz}{}
\[
\left [ \frac{\partial}{\partial T} \left ( \frac{G _ i}{T} \right ) \right ] _ p = - \frac{H _ i}{T ^ 2}
\]
\end{frame}

\begin{frame}{Cœfficients thermoélastiques}
\protect\hypertarget{cux153fficients-thermouxe9lastiques}{}
\begin{itemize}
\tightlist
\item
  Dilatation thermique isobare \[
  \alpha = \frac{1}{V} \left ( \frac{\partial V}{\partial T} \right ) _ P
  \]
\item
  Augmentation de pression isochore \[
  \beta = \frac{1}{P} \left ( \frac{\partial P}{\partial T} \right ) _ V
  \]
\item
  Compressibilité isotherme \[
  \chi _ T = - \frac{1}{V} \left ( \frac{\partial V}{\partial p} \right ) _ T
  \]
\end{itemize}
\end{frame}

\begin{frame}{Cœfficients calorimétriques}
\protect\hypertarget{cux153fficients-calorimuxe9triques}{}
\begin{itemize}
\tightlist
\item
  Les capacités calorifiques à volume et pression constantes sont
  définies par \[
  C _ {V i} = \left ( \frac{\partial U _ i}{\partial T} \right ) _ V
  \] et \[
  C _ {p i} = \left ( \frac{\partial H _ i}{\partial T} \right ) _ V
  \]
\item
  Elles s'expriment toutes deux en \(\si{\joule\per\kelvin\per\mole}\)
\end{itemize}
\end{frame}

\end{document}
